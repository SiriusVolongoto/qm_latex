\documentclass[a4paper]{article}

\usepackage{amsmath}
\usepackage{CJK}
\usepackage{graphicx}
\usepackage[top=30mm,bottom=22mm,left=25mm,right=25mm]{geometry}

\newcommand{\qed}{
    \rightline{Q.E.D.}\\
    }
\linespread{1.25}
\newcommand{\ft}{\footnotesize}
\newcommand{\dis}{\displaystyle}
\newcommand{\VEC}[1]{\overrightarrow{#1}}

\begin{document}
    %大标题
    \begin{center}
        \textsf{\LARGE{Quantum Mechanics (A) - Assignment 8}}\\[20pt]
    \end{center}
    % 签名 最后修改日期
    \begin{CJK*}{UTF8}{kai}
        \begin{flushright}
            \small PB12203077\quad吴奕涛\\
            \small Last Modified : \today\\[30pt]
        \end{flushright}
    \end{CJK*}
    % 正文
    % 作业布置日期
    \noindent\Large{\emph{October 30 , 2014}}\\[15pt]
    \begin{CJK*}{UTF8}{song}
    % 作业题 1
    \noindent 5.5 计算氢原子基态时,
        电子处于经典禁区$(r>2a)$(即$E-V<0$区域)的概率。\\[8pt]
    \textbf{解:}\\[5pt]
    {
    氢原子基态
    $$\psi=\psi_{100}(r,\theta,\varphi)=R_{10}Y_{00}
        =\frac{2}{a^{3/2}}e^{-r/a}\cdot\frac{1}{\sqrt{4\pi}}
        =\frac{1}{\sqrt{\pi}a^{3/2}}e^{-r/a}$$
    计算经典禁区内的概率
    $$P(r>2a)=\int_{0}^{2\pi}\mathrm{d\varphi}
        \int_{0}^{\pi}sin(\theta)\mathrm{d\theta}
        \int_{2a}^{\infty}r^{2}|\psi|^{2}\mathrm{dr}
        =\frac{13}{e^{4}}=0.23810$$
    }\\[20pt]
    \noindent 5.6 对于类氢原子(核电荷Ze)的“圆轨道”$(n_{r}=0)$,计算\\[8pt]
    \textbf{解:}\\[5pt]
    {
    (a)最概然半径。
    由波函数球对称性可知,在$(r,r+\mathrm{d}r)$球壳内找到电子的概率为
    $$P(r)\mathrm{d}r=r^{2}\mathrm{d}r\int\mathrm{d}\Omega|\psi|^{2}
        =|R_{nl}|^{2}r^{2}\mathrm{d}r$$
    故最概然半径$r_{m}$处有$\dis\frac{\mathrm{d}P(r)}{\mathrm{d}r}=0$,
        亦即$\dis\frac{\mathrm{d}(r R_{nl})}{\mathrm{d}r}=0$。\\
    对于类氢原子圆轨道基态
    $$r R_{nl}(r)=r R_{n,n-1}(r)=C r^{n}e^{-\frac{Z r}{n a}},\mbox{则}r_{m}=n^{2}a/Z$$
    (b)计算$\bar{r}$。\\
    令$\chi_{nl}=r R_{nl}$,则
    $$\bar{r}=\int_{V}r|\psi_{nlm}|^{2}\mathrm{d}^{3}r
        =\int_{0}^{\infty}r|\chi_{nl}|^{2}\mathrm{d}r$$
    已知$\chi_{nl}$满足微分方程
    $$\chi_{nl}^{''}+[\frac{2\mu}{\hbar^{2}}(E+\frac{Z e^{2}}{r})
        -\frac{l(l+1)}{r^{2}}]\chi_{nl} = 0$$
    将$r^{\lambda}\chi_{nl}$和$2r^{\lambda+1}\chi_{nl}$分别乘入上式并积分整理得
    $$\frac{\lambda +1}{\hbar^{2}}\overline{r^{\lambda}}
        -(2\lambda +1)\frac{a}{Z}\overline{r^{\lambda-1}}
        +\frac{\lambda}{4}[(2l+1)^{2}-\lambda^{2}](\frac{a}{Z})^{2}
        \overline{r^{\lambda-2}}=0$$
    上式即为Kramers定理。\\
    令$\lambda=0$得$\overline{r^{-1}}=\frac{Z}{n^{2}a}$,再令$\lambda=1$,即可得
    $$\bar{r}=n(n+\frac{1}{2})\frac{a}{Z}\quad,\quad l=n-1$$
    (c)计算$\Delta r$。\\
    利用(b)中结果,令$\lambda =2$可得
    $$\overline{r^{2}}=\frac{n^{2}}{2}(2 n+1)(n+1)(\frac{a}{Z})^{2}
        \quad,\quad l=n-1$$
    根据标准差计算公式
    $$\Delta r = \sqrt{\overline{r^{2}} - (\bar{r})^{2}}
        =\sqrt{\frac{n^{2}}{2}(2 n+1)(n+1)(\frac{a}{Z})^{2}}
        =\frac{n}{2}\sqrt{2 n+1}\frac{a}{Z}.$$
    }\\[20pt]
    \noindent 5.11 二维谐振子势$\dis V(x,y)=\frac{1}{2}m\omega_{x}^{2}x^{2}
        +\frac{1}{2}m\omega_{x}^{2}x^{2}$,设$\omega_{x}/\omega_{y}=1/2$,
        求能级的分布和简并度\\[8pt]
    \textbf{解:}\\[5pt]
    {
    给定的$V(x,y)$代入Hamilton量表达式中有
    $$\hat{H}=\frac{\hat{p^{2}}}{2m}+V(x,y)
        =-\frac{\hbar^{2}}{2 m}\frac{\partial^{2}}{\partial x^{2}}
        -\frac{\hbar^{2}}{2 m}\frac{\partial^{2}}{\partial x^{2}}
        +\frac{1}{2}m\omega_{x}^{2}x^{2}+\frac{1}{2}m\omega_{x}^{2}x^{2}$$
    设$\hat{H}=\hat{H_{x}}+\hat{H_{y}}$,其中:
    \begin{equation*}\begin{split}
    \hat{H_{x}} & =  -\frac{\hbar^{2}}{2 m}\frac{\partial^{2}}{\partial x^{2}}
        + \frac{1}{2}m\omega_{x}^{2}x^{2}\\
    \hat{H_{y}} & =  -\frac{\hbar^{2}}{2 m}\frac{\partial^{2}}{\partial y^{2}}
        + \frac{1}{2}m\omega_{y}^{2}y^{2}\\        
    \end{split}\end{equation*}
    显然有$\hat{H},\hat{H_{x}},\hat{H_{y}}$两两对易。\\
    利用一维谐振子结果可得$\hat{H_{x}},\hat{H_{y}}$本征值
    $$E_{n x}=(n_{x}+\frac{1}{2})\hbar\omega_{x}
        \quad,\quad E_{n y}=(n_{y}+\frac{1}{2})\hbar\omega_{y}$$
    设$\omega_{x}=\omega,\omega_{y}=2\omega$,则$\hat{H}$的能级分布为
    $$E_{n}=E_{n x}+E_{n y}=(n_{x}+2 n_{y} + \frac{3}{2})\hbar\omega
        \quad,\quad n = n_{x}+2 n_{y}$$
    分析能级简并度。\\
    当$n=2m+1$时,$(n_{x},n_{y})=(1,m),(3,m-1)...(2 m+1,0)$,
        简并度为m+1,即(n+1)/2。\\
    当$n=2m$时,$(n_{x},n_{y})=(0,m),(2,m-1)...(2 m,0)$,
        简并度为m+1,即n/2+1。\\
    }\\[20pt]
    \noindent\Large{\emph{November 3 , 2014}}\\[15pt]
    \noindent 随堂作业 - 证明:$\bar{A}=\langle\psi|A|\psi\rangle$\\[8pt]
    \textbf{解:}\\[5pt]
    {
    设A的本征态空间为$|a\rangle$,由此展开$|\psi\rangle = C_{a}|a\rangle$\\
    $$\langle\psi|A|\psi\rangle = \langle a|Ca\cdot A\cdot C_{a}|a\rangle
        = C_{a}^{2}\langle a|a|a\rangle = C_{a}^{2}a = \bar{A}$$
    \qed
    }\\[20pt]
    \end{CJK*}

\end{document} 