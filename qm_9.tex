\documentclass[a4paper]{article}

\usepackage{amsmath}
\usepackage{CJK}
\usepackage{graphicx}
\usepackage[top=30mm,bottom=22mm,left=25mm,right=25mm]{geometry}

\newcommand{\qed}{
    \rightline{Q.E.D.}\\
    }
\linespread{1.25}
\newcommand{\ft}{\footnotesize}
\newcommand{\dis}{\displaystyle}
\newcommand{\VEC}[1]{\overrightarrow{#1}}

\begin{document}
    %大标题
    \begin{center}
        \textsf{\LARGE{Quantum Mechanics (A) - Assignment 9}}\\[20pt]
    \end{center}
    % 签名 最后修改日期
    \begin{CJK*}{UTF8}{kai}
        \begin{flushright}
            \small PB12203077\quad吴奕涛\\
            \small Last Modified : \today\\[30pt]
        \end{flushright}
    \end{CJK*}
    % 正文
    % 作业布置日期
    \noindent\Large{\emph{November 10 , 2014}}\\[15pt]
    \begin{CJK*}{UTF8}{song}
    % 作业题 例1
    % 标题-黑体
    \begin{CJK*}{UTF8}{hei}
    \noindent 作业一、利用球极坐标下定态薛定谔方程验证单一能量的自由粒子的概率幅度正比于
    $\dis\frac{1}{r_{12}}e^{\mathrm{i}k r_{12}}$
    \end{CJK*}
    \\[12pt]
    {
    \noindent 证明:
    由题意可得$\dis\langle r_{2}|r_{1}\rangle = A e^{\mathrm{i} k r}$。\\
    自由粒子定态薛定谔方程的球坐标形式为
    $$-\frac{\hbar^{2}}{2 m}\nabla^{2}\psi
    = -\frac{\hbar^{2}}{2 m}[\frac{1}{r^{2}}\frac{\partial}{\partial r}(r^{2}\frac{\partial\psi}{\partial r})]=E\psi$$
    代入表达式可得
    \begin{equation*}\begin{split}
    \indent\emph{Left}
        & = -\frac{\hbar^{2}}{2 m}\frac{1}{r^{2}}\frac{\partial}{\partial r}(r^{2}\frac{\partial(A e^{\mathrm{i}k r}/r)}{\partial r})
         = -\frac{\hbar^{2}}{2 m}\frac{1}{r^{2}}\frac{\partial}{\partial r}(r^{2}
        \cdot A e^{\mathrm{i}k r}(\frac{\mathrm{i}k }{r}-\frac{1}{r^{2}}))\\
        & = -\frac{\hbar^{2}}{2 m}\frac{1}{r^{2}}\frac{\partial}{\partial r}
        (A e^{\mathrm{i}k r}(\mathrm{i}k r-1))
        = -\frac{\hbar^{2}}{2 m}\frac{1}{r^{2}}
        [\mathrm{i}k A e^{\mathrm{i}k r}(\mathrm{i}k r-1)+\mathrm{i}k A e^{\mathrm{i}k r}]\\
        & = \frac{\hbar^{2}k^{2}}{2 m}\frac{A e^{\mathrm{i}k r}}{r}\\
    \indent\emph{Right}
        & = E_{1}\psi = \frac{\hbar^{2}k^{2}}{2 m}\frac{A e^{\mathrm{i}k r}}{r}
    \end{split}\end{equation*}
    由此可得,Left=Right,原式成立。\\[8pt]
    \qed
    }\\[20pt]
    \begin{CJK*}{UTF8}{hei}
    \noindent作业二、假设光子线偏振态矢量分量$(\cos{\theta},\sin{\theta})^{T},\theta$为
    起偏器的角度、符号T表示转置,而对偶左矢量为右矢量的转置共轭,请验证实验结果出射强度
    $I=$入射强度$I_{0}\cos^{2}(\theta-\phi)$,其中$\phi$为检偏器的角度。
    \end{CJK*}
    \\[12pt]
    {
    \noindent 证明:\\
    根据光子线偏振态矢量分量定义,假设出射偏转角度为$\phi$,则出射振幅态
    $$\langle\phi |\theta\rangle
        =(\cos\phi \sin\phi)\binom{\cos\theta}{\sin\theta}
        =\cos\phi\cos\theta-\sin\phi-\sin\theta
        =\cos(\theta-\phi)
        $$
    由光强等于振幅的平方可得,$I=I_{0}\cos^{2}(\theta-\phi)$。\\[8pt]
    \qed
    }\\[20pt]
    \begin{CJK*}{UTF8}{hei}
    \noindent作业三、利用狄拉克符号的惠更斯原理对比傅立叶逆变换,推导
    $$\dis\langle k|x\rangle=\frac{1}{2\pi}e^{-\mathrm{i}k x}$$
    \end{CJK*}
    \noindent 证明:\\
    {
    已知$\dis\langle x|p\rangle=e^{\mathrm{i}p x/\hbar}$,
    即$\dis\langle x|k\rangle=e^{\mathrm{i}k x}$
    由惠更斯原理$\langle k | x \rangle = \langle x|k\rangle^{*}$\\
    对比Fourier逆变换
    $$\phi(k)=\frac{1}{2\pi}\int_{-\infty}^{\infty}\psi(x)e^{-\mathrm{i}k x}\mathrm{d}x$$
    即可得
    $$\dis\langle k|x\rangle=\frac{1}{2\pi}e^{-\mathrm{i}k x}$$
    \qed
    }\\[40pt]
    % 作业布置日期
    \noindent\Large{\emph{November 13 , 2014}}\\[15pt]
    % 标题-黑体
    \begin{CJK*}{UTF8}{hei}
    \noindent 作业一、在宽为L的无限深势阱中,质量m的粒子处于束缚态,试解决以下问题
    \end{CJK*}
    \noindent \emph{解:}\\
    {
    \begin{CJK*}{UTF8}{kai}
    (1)请给出能量本征态的x表示,即在x表象下的能量本征波函数$\langle x|E_{n} \rangle$。
    以及能量表象与坐标表象之间的变换式:\\
    \end{CJK*}
    根据波动方程形式的定态薛定谔方程无限深方势阱能量本征解可知
    $$\langle x|E_{n}\rangle = \psi_{n}(x)=\sqrt{\frac{2}{L}}\sin{\frac{n\pi x}{L}}$$
    已知$x$表象下,坐标本征态$\langle x|x^{'}\rangle =\delta(x-x')$,
    动量本征态$\langle x|p'\rangle = \frac{1}{\sqrt{2\pi\hbar}}e^{\mathrm{i}p'x/\hbar}$\\
    由$\hat{E}=\frac{\hat{p^{2}}}{2 m}+V(x)$可得,表象变换公式
    $$\langle E|\psi\rangle=\int\mathrm{d}x'\langle E|x'\rangle\langle x'|\psi\rangle
        =\int\mathrm{d}x'\langle x'|E\rangle^{*}\langle x'|\psi\rangle$$
    \begin{CJK*}{UTF8}{kai}
    \noindent(2)通解$|\psi\rangle=\sum_{n}|E_{n}\rangle C_{n}$中的在能量本征态下的展开系数的表达式
        $C_{n}=?\psi(x)$——用波函数$\psi(x)$构造。\\
    \end{CJK*}
    根据投影算符归一化定义$|\psi\rangle=\sum|E_{n}\rangle\langle E_{n}|\psi\rangle$,
    故由此可得$C_{n}=\langle E_{n}|\psi\rangle$\\
    \begin{CJK*}{UTF8}{kai}
    (3)逆表象变换表达式:波函数用展开系数的表达式$\psi(x)=?  C_{n}$。
    \end{CJK*}
    同(1)中变换方式可得
    $$\langle x|\psi\rangle = \sum\langle x|E_{n}\rangle\langle E_{n}|\psi\rangle 
        = \sum\langle x|E_{n}\rangle C_{n}$$
    \qed
    }\\[20pt]
    % 标题-黑体
    \begin{CJK*}{UTF8}{hei}
    \noindent 作业二、\emph{[曾谨言,教材习题7.5]}\\
    利用$x$表象和$p$表象之间的“幺正变换”,从7.3题推出7.4题中的结果
    \end{CJK*}
    \\[12pt]
    {
    \noindent \emph{解}:\\
    7.3题中的结果:$x$表象中$x,p,H$的"矩阵元"
    \begin{equation*}\begin{split}
    (x)_{x'x''} & = \langle x'|x|x''\rangle = x'\delta(x'-x'')\\
    (p)_{x'x''} & = \langle x'|\hat{p}|x''\rangle 
        = -\mathrm{i}\hbar\frac{\partial}{\partial x'}\delta(x'-x'')\\
    (H)_{x'x''} & = \langle x'|\hat{H}|x''\rangle
        =-\frac{{\hbar}^{2}}{2 m}\frac{\partial^{2}}{\partial {x'}^{2}}\delta(x'-x'')
        +V(x')\delta(x'-x'')\\
    \end{split}\end{equation*}
    $x$表象和$p$表象之间的“幺正变换”
    $$\langle x|x'\rangle =\delta(x-x')\quad,\quad
        \langle x|p'\rangle = \frac{1}{\sqrt{2\pi\hbar}}e^{\mathrm{i}p'x/\hbar}$$
    $$\langle p|p'\rangle =\delta(p-p')\quad,\quad
        \langle p|x'\rangle = \frac{1}{\sqrt{2\pi\hbar}}e^{-\mathrm{i}p x'/\hbar}$$        
    则由以上可得7.4中结果:$p$表象下$x,p,H$的“矩阵元”
    \begin{equation*}\begin{split}
    (x)_{p'p''} 
        & = \langle p'|x|p''\rangle
            = \iint\mathrm{d}x'\mathrm{d}x''\langle p'|x'\rangle\langle x'|x|x''\rangle
            \langle x''|p''\rangle\\
        & = \iint\mathrm{d}x'\mathrm{d}x''
            \frac{e^{-\mathrm{i}p'x'/\hbar}}{\sqrt{2\pi\hbar}}
            \cdot x'\delta(x'-x'')\cdot
            \frac{e^{\mathrm{i}p''x''/\hbar}}{\sqrt{2\pi\hbar}}\\
        & = \frac{1}{2\pi\hbar}\int\mathrm{d}x'\cdot x' e^{-\mathrm{i}(p'-p'')x'/\hbar}\\
        & = \frac{1}{2\pi\hbar}\cdot\mathrm{i}\hbar\frac{\partial}{\partial p'}
            \int\mathrm{d}x'\cdot e^{-\mathrm{i}(p'-p'')x'/\hbar}\\
        & = \mathrm{i}\hbar\frac{\partial}{\partial p'}\delta(p'-p'')\\
    (p)_{p'p''}
        & = \langle p'|p|p''\rangle
            = \iint\mathrm{d}x'\mathrm{d}x''\langle p'|x'\rangle\langle x'|p|x''\rangle
            \langle x''|p''\rangle\\
        & = \iint\mathrm{d}x'\mathrm{d}x''
            \frac{e^{-\mathrm{i}p'x'/\hbar}}{\sqrt{2\pi\hbar}}
            \cdot(-\mathrm{i}\hbar\frac{\partial}{\partial x'})\delta(x'-x'')\cdot
            \frac{e^{\mathrm{i}p''x''/\hbar}}{\sqrt{2\pi\hbar}}\\
        & = \frac{1}{2\pi\hbar}\int\mathrm{d}x'
            \cdot e^{-\mathrm{i}p'x'/\hbar}
            (-\mathrm{i}\hbar\frac{\partial}{\partial x'})
            e^{-\mathrm{i}p''x'/\hbar}\\
        & = \frac{1}{2\pi\hbar}\cdot p''
            \int\mathrm{d}x'\cdot e^{-\mathrm{i}(p'-p'')x'/\hbar}\\
        & = p'\delta(p'-p'')\\
    (H)_{p'p''}
        & = \langle p'|H|p''\rangle
            = \iint\mathrm{d}x'\mathrm{d}x''\langle p'|x'\rangle\langle x'|H|x''\rangle
            \langle x''|p''\rangle\\
        & = \iint\mathrm{d}x'\mathrm{d}x''
            \frac{e^{-\mathrm{i}p'x'/\hbar}}{\sqrt{2\pi\hbar}}
            \cdot [-\frac{\hbar^{2}}{2m}\frac{\partial^{2}}{\partial {x'}^{2}}
            \delta(x'-x'')+V(x')\delta(x'-x'')]\cdot
            \frac{e^{\mathrm{i}p''x''/\hbar}}{\sqrt{2\pi\hbar}}\\
        & = \frac{1}{2\pi\hbar}\int\mathrm{d}x'
            \cdot e^{-\mathrm{i}p'x'/\hbar}
            [-\frac{\hbar^{2}}{2m}\frac{\partial^{2}}{\partial {x'}^{2}}+V(x')]
            e^{\mathrm{i}p''x'/\hbar}\\
        & = \frac{1}{2\pi\hbar}\int\mathrm{d}x'\cdot
            [\frac{{p''}^{2}}{2m}+V(x')]e^{-\mathrm{i}(p'-p'')x'/\hbar}\\
        & = \frac{{p'}^{2}}{2m}\delta(p'-p'')+V(\mathrm{i}\hbar\frac{\partial}{\partial p'})\delta(p'-p'')
    \end{split}\end{equation*}
    }\\[20pt]
	% 作业布置日期
    \noindent\Large{\emph{November 17 , 2014}}\\[15pt]
    % 作业题 例1
    \begin{CJK*}{UTF8}{hei}
    \noindent 作业一、 证明:$\psi_{\alpha} = S_{\alpha i}\phi_{i}$\\[12pt]
    \end{CJK*}
    {
    \indent 已知$|\psi\rangle=\psi_{i}|i\rangle=\psi_{\alpha}|\alpha\rangle$,
        故若取$|\beta\rangle\in{|\alpha\rangle}$,则容易得到
    $$\langle\beta|\psi\rangle = \langle\beta|\psi_{i}|i\rangle=\psi_{i}\langle\beta|i\rangle$$
    $$\langle\beta|\psi\rangle = \langle\beta|\psi_{\alpha}|\alpha\rangle=\psi_{\beta}$$
    即$\psi_{\beta} = \langle\beta|i\rangle \psi_{i}= S_{\beta i}\psi_{i}$\\
    \qed
    }\\[20pt]
    \begin{CJK*}{UTF8}{hei}
    \noindent 作业二、 证明:$C|\psi\rangle\leftrightarrow\langle\psi|C^{*}$\\[12pt]
    \end{CJK*}
    {
    \indent 已知$\langle\varphi|\psi\rangle^{*}=\langle\psi|\varphi\rangle$,
        则$(\langle\varphi|C|\psi\rangle)^{*}=\langle\psi|C^{*}|\varphi\rangle$\\
    \indent 由$\rangle\varphi|\leftrightarrow|\varphi\rangle$可得
    $$C|\psi\rangle\leftrightarrow\langle\psi|C^{*}$$
    \qed
    }\\[20pt]
    \begin{CJK*}{UTF8}{hei}
    \noindent 作业三、 证明:$\widetilde{A}_{ij}=A_{ji}$\\[12pt]
    \end{CJK*}
    {
    \indent 根据转置算符的定义式
    $\langle\psi|\widetilde{A}|\varphi\rangle=\langle\varphi|A|\psi\rangle$,
    插入恒等算符可得
    $$\langle\psi|\widetilde{A}|\varphi\rangle 
        = \langle\psi|i\rangle\langle i|\widetilde{A}|j\rangle\langle j|\varphi\rangle
        =\psi_{i}\varphi_{j}\widetilde{A}_{ij}$$
    $$\langle\varphi|A|\psi\rangle
        = \langle\varphi|j\rangle\langle j|A|i\rangle\langle i|\varphi\rangle
        =\psi_{i}\varphi_{j}A_{ji}$$
    \indent 由以上二式可得
    $$\widetilde{A}_{ij}=A_{ji}$$ 
    }\\[20pt]
    \begin{CJK*}{UTF8}{hei}
    \noindent 作业四、 讨论$U,U',U^{+}$关系\\[12pt]
    \end{CJK*}
    {
    \indent 定义$U=S_{i\alpha}|i\rangle\langle\alpha|$,同作业三运算可得\\
    \begin{equation*}\begin{split}
    \langle\psi|U|\varphi\rangle
        & =\langle i|\alpha\rangle\langle\psi|i\rangle\langle\alpha|\varphi\rangle
            =\psi_{i}\varphi_{\alpha}\langle i|\alpha\rangle\\ 
    \langle\varphi|\widetilde{U}|\psi\rangle
        & = \langle\varphi|\alpha\rangle\langle\alpha|
            \widetilde{U}|i\rangle\langle i|\psi\rangle
            =\psi_{i}\varphi_{\alpha}\langle\alpha|\widetilde{U}|i\rangle\\
    \end{split}\end{equation*}
    $$\widetilde{U}
        = |\alpha\rangle\langle i|\alpha\rangle\langle i|
        = \langle\alpha|i\rangle|\alpha\rangle\langle i|
        =S_{\alpha i}|\alpha\rangle\langle i|$$
    \indent 相应地,由厄米转置定义
        $\langle\psi|A^{+}|\varphi\rangle=\langle A\psi|\varphi\rangle$可得
    $$\langle\psi|U^{+}|\varphi\rangle = (\langle\varphi|U|\psi\rangle)^{*}
        =\langle\psi|\widetilde{U}^{*}|\varphi\rangle$$
    \indent 即直接写出$U^{+}$表达式
    $$U^{+}=\widetilde{U}^{*}=S_{i\alpha}|\alpha\rangle\langle i|$$
    }\\[20pt]
    \end{CJK*}
\end{document} 