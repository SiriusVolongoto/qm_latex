\documentclass[a4paper]{article}

\usepackage{amsmath}
\usepackage{CJK}
\usepackage{graphicx}
\usepackage[top=30mm,bottom=22mm,left=25mm,right=25mm]{geometry}

\newcommand{\qed}{
    \rightline{Q.E.D.}\\
    }
\linespread{1.25}
\newcommand{\ft}{\footnotesize}
\newcommand{\dis}{\displaystyle}
\newcommand{\VEC}[1]{\overrightarrow{#1}}

\begin{document}
    %大标题
    \begin{center}
        \textsf{\LARGE{Quantum Mechanics (A) - Assignment 6}}\\[20pt]
    \end{center}
    % 签名 最后修改日期
    \begin{CJK*}{UTF8}{kai}
        \begin{flushright}
            \small PB12203077\quad吴奕涛\\
            \small Last Modified : \today\\[30pt]
        \end{flushright}
    \end{CJK*}
    % 正文
    \begin{CJK*}{UTF8}{song}
    % 作业布置日期
    \noindent\Large{\emph{October 23 , 2014}}\\[15pt]
    % 作业题 例1
    \begin{CJK*}{UTF8}{kai}
    \noindent 随堂1\quad 计算$[\hat{L_{i}},\hat{p}_{j}]$\\[12pt]
    \end{CJK*}
    {
    \textbf{解:}
    \begin{equation*}
    \begin{split}
    [\hat{L_{i}},\hat{p_{j}}]
        & = [(x_{i+1}\hat{p}_{i-1}-x_{i-1}\hat{p}_{i+1}),\hat{p_{j}}]\\
        & = [x_{i+1}\hat{p}_{i-1},\hat{p_{j}}]-[x_{i-1}\hat{p}_{i+1}),\hat{p_{j}}]\\
        & = x_{i+1}[\hat{p}_{i-1},\hat{p_{j}}]+[x_{i+1},\hat{p_{j}}]\hat{p}_{i-1}
            - x_{i-1}[\hat{p}_{i+1},\hat{p_{j}}] - [x_{i-1},\hat{p_{j}}]\hat{p}_{i+1}\\
        & = [x_{i+1},\hat{p_{j}}]\hat{p}_{i-1} - [x_{i-1},\hat{p_{j}}]\hat{p}_{i+1}
    \end{split}
    \end{equation*}
    对j进行分类讨论:
    \[[\hat{L_{i}},\hat{p_{j}}] =\left\{
    \begin{array}{ll}
        -\mathrm{i}\hbar \hat{p}_{i+1}\quad &,\quad j = i -1\\
        0 \quad & ,\quad j = i \\
        \mathrm{i}\hbar \hat{p}_{i-1} & , \quad j = i + 1
    \end{array}
    \right.
    \]
    利用Levi-Civita符号整理即
    $$[\hat{L_{i}},\hat{p}_{j}] = \epsilon_{ijk}\mathrm{i}\hbar \hat{p}_{k}.$$
    }\\[20pt]
    \noindent 3.14 证明$\overline{L_{x}}=\overline{L_{y}}=0$.\\[8pt]
    {
    \textbf{证明:}
    \begin{equation*}
    \begin{split}
    \overline{L_{x}}
        & = \int\psi^{*}(\vec{r})\hat{L}_{x}\psi(\vec{r})\mathrm{d\vec{r}}\\
        & = -\frac{\mathrm{i}}{\hbar}\int\psi^{*}(\vec{r})
           [\hat{L}_{y},\hat{L}_{z}]\psi(\vec{r})\mathrm{d\vec{r}}\\
        & = -\frac{\mathrm{i}}{\hbar}\int[\psi^{*}(\vec{r})\hat{L}_{y}(\hat{L}_{z}\psi(\vec{r}))
            -(\hat{L}_{z}\psi^{*}(\vec{r}))\hat{L}_{y}\psi(\vec{r})]\mathrm{d\vec{r}}\\
        & = -\frac{\mathrm{i}}{\hbar}\int[\psi^{*}(\vec{r})\hat{L}_{y}(m\hbar\psi(\vec{r}))
            -(m\hbar\psi^{*}(\vec{r}))\hat{L}_{y}\psi(\vec{r})]\mathrm{d\vec{r}}\\
        & = 0            
    \end{split}
    \end{equation*}
    同理可得,$\overline{L_{x}}=0$。\\
    \qed
    }\\[20pt]    
    \noindent 3.16 设体系处于$\psi = c_{1}Y_{11}+c_{2}Y_{20}$状态。试计算:\\[8pt]
    {
    \textbf{解:}\\[5pt]
    (a)$L_{z}$的可能测值及期望值:\\
    根据本征解性质,已知$\hat{L_{z}}\psi = m\hbar\psi$,故
    $$\hat{L_{z}}Y_{11}=\hbar Y_{11}\quad,\quad\hat{L_{z}}Y_{20}=0\cdot\hbar Y_{20}$$
    即可能的观测值为$\hbar,2\hbar$,相应的概率为$|c_{1}|^{2},|c_{2}^{2}|$,得
    $$\overline{L_{z}}=|c_{1}|^{2}\hbar.$$
    (b)$\vec{L}^{2}$的可能测值及相应的概率:\\
    已知$\hat{\vec{L}}^{2}\psi = l(l+1)\hbar^{2}\psi$,故
    $$\hat{\vec{L}}^{2}Y_{11}=2\hbar^{2} Y_{11}\quad,
        \quad\hat{\vec{L}}^{2}Y_{20}=6\hbar^{3} Y_{20}$$
    即可能的观测值为$2\hbar^{2},6\hbar^{2}$,相应的概率为$|c_{1}|^{2},|c_{2}^{2}|$。\\[5pt]
    (c)$L_{x}$的可能测值及相应的概率:\\
    根据x,y,z轮换不变性可知$L_{x}$ 的可能取值为 $0,\pm\hbar,...,\pm l\hbar$。\\
    以下在$(\hat{\vec{L}}^{2},\hat{L}_{z})$本征解(基矢)空间$\{Y_{lm}\}$中
        求$\hat{L}_{x}$本征解$\{\phi_{lm}\}$:\\
    1.(l=1)基矢为$\{Y_{10},Y_{1,\pm1}\}$\\
    \indent 查阅教材附录得到具体形式
    $$\{\sqrt{\frac{3}{4\pi}}\frac{z}{r}
        ,\mp\sqrt{\frac{3}{8\pi}}\frac{x\pm \mathrm{i}y}{r}\}$$
    \indent 作$x\rightarrow y,y\rightarrow z,z\rightarrow x$轮换得到
    $$\phi_{10}=\frac{1}{\sqrt{2}}(Y_{11}-Y_{1,-1})\quad,
        \quad\phi_{1,\pm1}=\frac{1}{2}(Y_{11}\pm \sqrt{2}Y_{10}+Y_{1,-1})$$
    \indent 由此可得$L_{x}$在$c_{1}Y_{11}$状态下可能值为$-\hbar,0,\hbar$,
        相应的概率为$|c_{1}|^{2}/4,|c_{2}^{2}|/2,|c_{1}|^{2}/4$\\
    2.($l$=2)基矢为$\{Y_{20},Y_{2,\pm1},Y_{2,\pm2}\}$\\
    \indent 查阅教材附录得到具体形式
    $$\{\sqrt{\frac{5}{16\pi}}\frac{2 z^{2}- x^{2}- y^{2}}{r^{2}}
        ,\mp\sqrt{\frac{15}{8\pi}}\frac{(x\pm \mathrm{i}y)z}{r^{2}}
        ,\frac{1}{2}\sqrt{\frac{15}{8\pi}}(x\pm \mathrm{i}y)^{2}\}$$
    \indent 作$x\rightarrow y,y\rightarrow z,z\rightarrow x$轮换得到
    \begin{equation*}\begin{split}
    \phi_{20} 
        & = \sqrt{\frac{3}{8}}(Y_{22}-\sqrt{\frac{2}{3}}Y_{20}+Y_{2,-2})\\
    \phi_{2,1} 
        & = \frac{1}{2}(Y_{22}+Y_{21}-Y_{2,-1}-Y_{2,-2})\\
    \phi_{2,-1}
        & = \frac{1}{2}(Y_{22}-Y_{21}+Y_{2,-1}-Y_{2,-2})\\
    \phi_{2,\pm 2}
        & = \frac{1}{4}(Y_{22}\pm2Y_{21}\mp\sqrt{6}Y_{20}\pm 2Y_{2,-1}+Y_{2,-2})      
    \end{split}\end{equation*}
    \indent 由此可得$L_{x}$在$c_{2}Y_{20}$状态下可能值为$-2\hbar-\hbar,0,\hbar,2\hbar$,
        相应的概率为$\{\displaystyle\frac{3}{8},0,\frac{1}{4},0,\frac{3}{8}\}|c_{2}|^{2}$\\
    综上所得,$L_{x}$在状态$\psi$下可能的测量值及其概率为
    $$
    P(L_{x}=0)=\frac{1}{2}|c_{1}|^{2}+\frac{1}{4}|c_{2}|^{2},
        P(\pm\hbar)=\frac{1}{4}|c_{1}|^{2},P(\pm2\hbar)=\frac{3}{8}|c_{2}|^{2}
        $$
    }\\[20pt]
    \noindent \textbf{附加题:}\quad 计算$L_{y}$在状态$\psi = Y_{10}|_{z}$下的可能测量值与概率\\[8pt]
    {
    \noindent\textbf{解:}\\
    类似3.16(c)$l=1$的情况,
        作$x\rightarrow z,y\rightarrow x,z\rightarrow y$轮换得到
    $$\phi_{10}=\frac{1}{\sqrt{2}}(Y_{11}+Y_{1,-1})\quad,
        \quad\phi_{1,\pm1}=\frac{1}{2}(Y_{11}\pm \mathrm{i}\sqrt{2}Y_{10}-Y_{1,-1})$$
    由此可得
    $$P(L_{y}=0)=0\quad,\quad P_{L_{y}=\pm\hbar}=\frac{1}{2}.$$
    }\\[40pt]     
    \noindent\Large{\emph{October 27 , 2014}}\\[15pt]
    \noindent \textbf{5.1:证明质心系坐标与实验室坐标力学量间的变换关系}\\[8pt]
    {
    \noindent\textbf{证明:}\\[5pt]
    质心系坐标基本关系
    $$\mbox{质心坐标}\vec{R}
        =\frac{m_{1}\vec{r}_{1}+m_{2}\vec{r}_{2}}{m_{1}+m_{2}}\quad,
        \quad\mbox{相对坐标}\vec{r}=\vec{r}_{1}-\vec{r}_{2}$$
    $$\mbox{体系总质量}M=m_{1}+m_{2}\quad,
        \mbox{体系折合质量}\mu=\frac{m_{1}m_{2}}{m_{1}+m_{2}}$$
    相对动量
    \begin{equation*}\begin{split}
    \vec{p} & = \mu\dot{\vec{r}}=\frac{m_{1}m_{2}}{m_{1}+m_{2}}
        (\dot{\vec{r_{1}}}-\dot{\vec{r_{2}}})
        =\frac{m_{1}m_{2}}{m_{1}+m_{2}}(\vec{p}_{1}/m_{1}-\vec{p}_{2}/m_{2})\\
        & =\frac{1}{M}(m_{2}\vec{p}_{1}-m_{1}\vec{p}_{2})
    \end{split}\end{equation*}
    总动量
    $$\vec{P}=M\dot{\vec{R}}=(m_{1}+m_{2})
        \frac{m_{1}\dot{\vec{r}}_{1}+m_{2}\dot{\vec{r}}_{2}}{m_{1}+m_{2}}
        =\vec{p}_{1}+\vec{p}_{2}$$
    总轨道角动量
    \begin{equation*}\begin{split}
    \vec{L} 
        & = \vec{R}\times\vec{P}+\vec{r}\times\vec{p}\\
        & = \frac{m_{1}\vec{r}_{1}+m_{2}\vec{r}_{2}}{m_{1}+m_{2}}
            \times(\vec{p}_{1}+\vec{p}_{2})+(\vec{r}_{1}-\vec{r}_{2})
            \times\frac{m_{2}\vec{p}_{1}-m_{1}\vec{p}_{2}}{m_{1}+m_{2}}\\
        & = \vec{r_{1}}\times\vec{p_{1}}+\vec{r_{2}}\times\vec{p_{2}}\\
        & = \VEC{l_{1}}+\VEC{l_{2}}    
    \end{split}\end{equation*}
    总能量
    \begin{equation*}\begin{split}
    T   & = \frac{\vec{P}^{2}}{2M}+\frac{\vec{p}^{2}}{2\mu}\\
        & = \frac{(\vec{p}_{1}+\vec{p}_{2})^{2}}{2(m_{1}+m_{2})}
            +\frac{(m_{2}\vec{p}_{1}-m_{1}\vec{p}_{2})^{2}}               {2(m_{1}+m_{2})^{2}m_{1}m_{2}/(m_{1}+m_{2})}\\
        & =\frac{\vec{p}_{1}^{2}}{2 m_{1}}+\frac{\vec{p}_{2}^{2}}{2 m_{2}}
    \end{split}\end{equation*}
    坐标反变换
    $$\vec{R} + \frac{\mu}{m_{1}}\vec{r} = 
        \frac{m_{1}\vec{r}_{1}+
        m_{2}\vec{r}_{2}}{m_{1}+m_{2}}+\frac{m_{2}}{m_{1}+m_{2}}
        (\vec{r}_{1}-\vec{r}_{2})
        =\vec{r}_{1}$$
    $$\vec{R} - \frac{\mu}{m_{2}}\vec{r} =
        \frac{m_{1}\vec{r}_{1}+m_{2}\vec{r}_{2}}{m_{1}+m_{2}}
        -\frac{m_{1}}{m_{1}+m_{2}}(\vec{r}_{1}-\vec{r}_{2})
        =\vec{r}_{2}$$
    $$\frac{\mu}{m_{2}}\vec{P}+\vec{p}
        = \frac{m_{1}}{m_{1}+m_{2}}(\vec{p}_{1}+\vec{p}_{2})
        +\frac{m_{2}\vec{p}_{1}-m_{1}\vec{p}_{2}}{m_{1}+m_{2}}
        =\vec{p}_{1}$$
    $$\frac{\mu}{m_{1}}\vec{P}-\vec{p}
        = \frac{m_{2}}{m_{1}+m_{2}}(\vec{p}_{1}+\vec{p}_{2})
        -\frac{m_{2}\vec{p}_{1}-m_{1}\vec{p}_{2}}{m_{1}+m_{2}}
        =\vec{p}_{2}$$
    \qed        
    }\\[20pt]
    \noindent \textbf{5.3:利用氢原子能级公式,讨论下列体系的能谱}\\[8pt]
    {
    \noindent\textbf{解:}\\[5pt]
    氢原子能级公式$\dis E_{n}=-\frac{\mu e^{4}}{2\hbar^{2}}\frac{1}{n^{2}}$\\
    }\\[20pt]
    \noindent \textbf{随堂证明:}\quad
        $\hat{L_{z}}=-\mathrm{i}\hbar\frac{\partial}{\partial\varphi}$\\[8pt]
    {
    \noindent\textbf{证明:}\\[5pt]
    已知直角坐标与球坐标变换关系
    \[\left\{
    \begin{array}{ll}
        x & = r\mathrm{sin}\theta\mathrm{cos}\varphi\\
        y & = r\mathrm{sin}\theta\mathrm{sin}\varphi\\
        z & = r\mathrm{cos}\theta\\\
    \end{array}
    \right.
    \quad,\quad
    \left\{
    \begin{array}{ll}
        r   & = \sqrt{x^{2}+y^{2}+z^{2}}\\
        \theta
            & = \mathrm{arctan}(\sqrt{x^{2}+y^{2}}/z)\\
        \varphi
            & = \mathrm{arctan}(y/x)
    \end{array}
    \right.
    \]
    简单微分运算易得
    $$\frac{\partial r}{\partial x}=\frac{x}{r},
        \frac{\partial\theta}{\partial x}=\frac{xz}{r^{2}\sqrt{x^{2}+y^{2}}},
        \frac{\partial\varphi}{\partial x}=-\frac{y}{x^{2}+y^{2}}$$
    $$\frac{\partial r}{\partial y}=\frac{y}{r},
        \frac{\partial\theta}{\partial y}=\frac{yz}{r^{2}\sqrt{x^{2}+y^{2}}},
        \frac{\partial\varphi}{\partial y}=\frac{x}{x^{2}+y^{2}}$$
    根据多变量函数微分链式法则
    $$\frac{\partial}{\partial x}=
        \frac{\partial}{\partial r}(\frac{\partial r}{\partial x})
        +\frac{\partial}{\partial\theta}(\frac{\partial\theta}{\partial x})
        +\frac{\partial}{\partial\varphi}(\frac{\partial\varphi}{\partial x})$$
    $$\frac{\partial}{\partial y}=
        \frac{\partial}{\partial r}(\frac{\partial r}{\partial y})
        +\frac{\partial}{\partial\theta}(\frac{\partial\theta}{\partial y})
        +\frac{\partial}{\partial\varphi}(\frac{\partial\varphi}{\partial y})$$
    代入直角坐标下的动量分量表达式
        $\displaystyle\hat{L}_{z}=-\mathrm{i}\hbar
        (x\frac{\partial}{\partial y}-y\frac{\partial}{\partial x})$可得
    $$\hat{L_{z}}=-\mathrm{i}\hbar\frac{\partial}{\partial\varphi}$$.
    \qed    
    }\\[20pt]                
    \end{CJK*}

\end{document} 