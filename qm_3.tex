\documentclass[a4paper,11pt]{article}
\usepackage{CJK}
\begin{document}

    \LARGE \textbf{\textrm{QM-3}}\\[3pt]
    \begin{CJK*}{UTF8}{kai}
    \section{Monday}
    随堂作业- 证明:\quad$\overline{\textbf{p}} = (\psi , \widetilde{\textbf{p}}\psi)
    = \displaystyle\int \textbf{p}|\psi (\textbf{p})|^{2}\mathrm{d}\textbf{p}$\\[6pt]
    \end{CJK*}
    
    \begin{CJK*}{UTF8}{song}
    \noindent 证明: 已知
    $$\overline{\textbf{p}} = \int \psi ^{*} (\textbf{r})\hat{\textbf{p}}\psi (\textbf{r})\mathrm{d^{3} r}$$
    将算符$\displaystyle \hat{\textbf{p}} = -\mathrm{i} \hbar \nabla$
    \end{CJK*}    
    
    \section{Thursday}
    \begin{CJK*}{UTF8}{kai}
        \begin{flushright}
            \small PB12203077\quad吴奕涛\\
            \footnotesize \today
        \end{flushright}
    \end{CJK*}
\end{document} 