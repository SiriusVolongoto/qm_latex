\documentclass[a4paper]{article}

\usepackage{amsmath}
\usepackage{CJK}
\usepackage{graphicx}
\usepackage[top=30mm,bottom=22mm,left=25mm,right=25mm]{geometry}

\newcommand{\qed}{
    \rightline{Q.E.D.}\\
    }
\linespread{1.25}

\begin{document}
    %大标题
    \begin{center}
        \textsf{\LARGE{Quantum Mechanics (A) - Assignment 6}}\\[20pt]
    \end{center}
    % 签名 最后修改日期
    \begin{CJK*}{UTF8}{kai}
        \begin{flushright}
            \small PB12203077\quad吴奕涛\\
            \small Last Modified : \today\\[30pt]
        \end{flushright}
    \end{CJK*}
    % 正文
    % 作业布置日期
    \noindent\Large{\emph{October 16 , 2014}}\\[15pt]
    \begin{CJK*}{UTF8}{song}
    % 作业题 例1
    \noindent 2.7 \textbf{证明:}\\[12pt]
    {
    (a)
    \begin{eqnarray*}
        \emph{Left}  & = &\\
        \emph{Right} & = & 
    \end{eqnarray*}
    由此可得,Left=Right,原式成立。\\[8pt]
    \qed
    }\\[20pt]
    \end{CJK*}

\end{document} 