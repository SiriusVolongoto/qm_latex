
\documentclass[a4paper,10pt,twoside]{cpc-hepnp}

\usepackage{multicol}
\usepackage{graphicx}
\usepackage{booktabs}
\usepackage{amssymb,bm,mathrsfs,bbm,amscd}
\usepackage[tbtags]{amsmath}
\usepackage{lastpage}
\usepackage{CJK}


\begin{document}
\begin{CJK*}{GBK}{song}

% ҳüҳ��
\fancyhead[c]{\small Chinese Physics C~~~Vol. 37, No. 1 (2013)
010201} 
\fancyfoot[C]{\small 010201-\thepage}
% ��ҳ��ע
\footnotetext[0]{Received 14 March 2009}



% ����
\title{Instruction for typesetting manuscripts\thanks{Supported by National Natural Science
Foundation of China (55555555) }}

\author{%
      Author1$^{1,2;1)}$\email{wuyf@mail.ihep.ac.cn}%
\quad WANG Xiao-Jun(����)$^{2;2)}$\email{hepnp@mail.ihep.ac.cn}%
\quad LI Da-Ming(�����)$^{1}$
\quad F. Jone$^{2}$
}
\maketitle


\address{%
$^1$ Institution or University where the author works,  district,  postal code,  country\\
$^2$ {\bf Example}: Institute of High Energy Physics, Chinese Academy of Sciences, Beijing 100049, China\\
}


\begin{abstract}
The abstract should summarize the context, content
and conclusions of the paper in less than 200 words. It should
not contain any references or displayed equations. Typeset the
abstract in 8~pt roman, making
an indentation of 1.5 pica on the left and right margins.
\end{abstract}


\begin{keyword}
keyword,  3--8 words separated by comma
\end{keyword}

\begin{pacs}
1--3 PACS(Physics and Astronomy Classification Scheme, http://www.aip.org/pacs/pacs.html/)
\end{pacs}


\footnotetext[0]{\hspace*{-3mm}\raisebox{0.3ex}{$\scriptstyle\copyright$}2013
Chinese Physical Society and the Institute of High Energy Physics
of the Chinese Academy of Sciences and the Institute
of Modern Physics of the Chinese Academy of Sciences and IOP Publishing Ltd}%


\begin{multicols}{2}



\section{Introduction}


This template is made on the basis of article.cls after
modifications. The character size is set to 10pt and the paper size
is A4. See cpc-hepnp.cls for the corresponding setup\cite{lab1}.


Some ten macro packages are used in this template\cite{lab1,lab2,lab3}. All of these macro
packages are built-in Medium tex, and do not need to be downloaded
or installed. Latex 2.3 or even higher versions can be used without
any problem. You can add more macro packages by yourself in case of
need while writing, but attention should be paid to the possible
conflicts of macro packages.

You can freely use a variety of tex instructions or redefine some of
their instructions to achieve your desired results. However, we
still have some special regulations which are mentioned in the
following.


\section{Title and author}


Funding information is put behind the title, use $ \backslash
$thanks\(\) command. When making an entry of the author's name,
attention should be paid to the emblem of the author's institution.
Different to the command in revtex 4, ``Chinese Physics C" uses
$\backslash $danwei\(\) command in writing the author's institution.
To avoid a mistake at this point, more attention should be called
for.

\section{Major headings}

Major headings should be typeset in boldface, and only the first
letter of the first word is capitalized.

\subsection{Sub-headings}

Sub-headings should be typeset in boldface and only the first letter
of the first word is capitalized. Section number is in boldface
Roman.

\subsubsection{Sub-subheadings}

Sub-subheadings are typeset in normal face and only the first letter
of the first word is capitalized. Section number is in Roman.


\section{Equations}


Equations can be written by using the normal equation environment
\begin{eqnarray}
\label{eq2}
w(q,t;q_{0},0) &=& \frac{1}{ \sqrt{2\pi}}
\frac{\omega}{ \sqrt{ T  ({\rm e}^{\frac{2\omega^{2}}{\bm\beta}  t}-1)}}\times\nonumber\\[1mm]
&&
\exp \left\{ -\frac{1}{2}\frac{[q- \langle q(t)\rangle ]^{2}\omega^{2}}{T({\rm e}^{\frac{2\omega^2}{\bm\beta} t}-1)}
\right\}.
\end{eqnarray}
In writing an equation one should be aware of the following points:

1) At the place where the equation has to change to another line,
the operator $\backslash\backslash$ should be placed.

2) Fractions should be expressed in a big fractional form with
the command $\backslash$dfrac\{\}\{\}.

3) Punctuation marks should be placed at the end of each formula.

For a non-breakable long equation, the banner layout should be used
with commands $\backslash $end\{multicols\} and $\backslash
$begin\{multicols\}\{2\}
\end{multicols}
%\ruleup
\begin{equation}
\label{one}
p(t;q_{10},q_{20}) = \vint\nolimits_{0}^{\;\infty} {\rm d} q_{1}
\vint\nolimits_{-1}^{+1} {\rm d} q_{2}w(q_{1},q_{2},t;q_{10},q_{20})=
\frac{1}{4} {\rm e}^{\eta t}\frac{1}{\sqrt{{\rm Det}{\cal A}(t)}}[1+{\rm Erf}(z_{1})][1+{\rm Erf}(z_{2})].
\end{equation}
%\ruledown \vspace{0.5cm}
\begin{multicols}{2}
For easy reading, commands $\backslash$ruleup and
$\backslash$ruledown are also used. This banner layout is applicable
for graphics and tables as well.

Moreover, equations should be referred to in abbreviated form,
e.g.~``Eq.~(\ref{eq2})'' or ``(2)''. In multiple-line equations, the
number should be given on the last line.


\section{Figures}

Figures should be inserted in the text nearest their first
reference. For a figure whose size does not exceed half the width of
a column as shown in Fig.~\ref{fig1}, the $ \backslash
$begin\{center\} and $ \backslash $end\{center\} environment should
be used.
\begin{center}
\includegraphics[width=4cm]{cpcf1.eps}
\figcaption{\label{fig1}   Figure 1. }
\end{center}
If the figure is fairly large, the banner layout shown in
Fig.~\ref{fig2} should be applied.
\end{multicols}
\ruleup
\begin{center}
\includegraphics[width=12cm]{cpcf2.eps}
\figcaption{\label{fig2} Figure 2.}
\end{center}
\ruledown


\begin{multicols}{2}


\section{Tables}

Figures should be placed in the text as close to the point of
reference as possible. The commands for inserting tables are the
same as those for figures. The table should be written in a
three-line form shown in Table~\ref{tab1}. The banner layout should
also be used for a wide table shown in Table~\ref{tab2}.


\begin{center}
\tabcaption{ \label{tab1}  Narrow table.}
\footnotesize
\begin{tabular*}{80mm}{c@{\extracolsep{\fill}}ccc}
\toprule Mass & $\sigma$/mb   & $\rho$  & \% Error \\
\hline
1.0\hphantom{00} & \hphantom{0}281.0 & \hphantom{0}280.81 & 0.07 \\
0.1\hphantom{00} & \hphantom{0}876.0 & \hphantom{0}875.74 & 0.03 \\
0.01\hphantom{0} & 2441.0 & 2441.0\hphantom{0} & 0.0\hphantom{0} \\
0.001 & 4130.0 & 4129.3\hphantom{0} & 0.17\\
0.0001 & 6130.0 & 6128.3\hphantom{0} & 0.28\\
\bottomrule
\end{tabular*}
\end{center}

\end{multicols}



\begin{center}
\tabcaption{ \label{tab2}  Wide table.}
\footnotesize
\begin{tabular*}{170mm}{@{\extracolsep{\fill}}ccccccc}
\toprule $l$/mm & $t_1$/s  & $t_2$/s & \% Error & Frequency/(rad/s) &  Model/(rad/s)   &$\beta$/fm$^2$\\
\hline
1.0\hphantom{00} & \hphantom{0}281.0 & \hphantom{0}280.81 & 0.07 & 37.6 & $-$0.297 & 0.17\\
0.1\hphantom{00} & \hphantom{0}876.0 & \hphantom{0}875.74 & 0.03 & 37.7 & $-$0.307 & 0.17\\
0.01\hphantom{0} & 2441.0 & 2441.0\hphantom{0} & 0.0\hphantom{0} & 37.4 & $-$0.283 & 0.17\\
0.001 & 4130.0 & 4129.3\hphantom{0} & 0.17& 37.5 & $-$0.290 & 0.17\\
\bottomrule
\end{tabular*}%
\end{center}

\begin{multicols}{2}




\section{References}

Journal: author's name. journal name, year of publication, volume
number (Issue No.): page number (as shown in Ref.~\cite{lab1})

Monographs: the author's name. Title. Version (version 1 can be
abbreviated). Published in: Publisher, publication year. Page No.
(The format is shown as in Ref.~\cite{lab2})

Collection: the author's name. Text title. See (in English ):
Editor. Essays name. Published in: Publisher, Publication year. Page
No. (The format is shown as in Ref.~\cite{lab3})


In the text, commands $\backslash $cite\{lab1\} or $\backslash
$cite\{lab1, lab2, lab3\} is used for citing a single
reference or a number of references.




\section{Footnotes}

Footnotes should be numbered sequentially in superscript
lowercase roman letters.\footnote{Footnotes should be
typeset in 8~pt  roman at the bottom of the page.}
\\



\acknowledgments{We thank  $\cdots$.}

\end{multicols}

\vspace{10mm}




\begin{multicols}{2}

\subsection*{Appendices A}
\begin{small}

\noindent{\bf Subtitle}

Appendices are generally placed after the references. The equations
should be numbered as A1, A2, $ \cdots $, and the letter size of the
text should be 9pt.


\begin{subequations}
\renewcommand{\theequation}{A\arabic{equation}}
\begin{equation}
\mu(n, t) = {\sum^\infty_{i=1} 1(d_i < t, N(d_i)
= n)}{\vint\nolimits^{\;t}_{\sigma=0} 1(N(\sigma) = n){\rm d}\sigma}\,. \label{a1}
\end{equation}

\begin{equation}
\mu(n, t) = {\sum^\infty_{i=1} 1(d_i < t,
N(d_i)=n)}{\vint\nolimits^{\;t}_{\sigma=0} 1(N(\sigma) = n){\rm d}\sigma}\,. \label{a2}
\end{equation}

\begin{equation}
\label{a3}
F = S \prod \limits_{i < j} \Big \{ \sum_{p=1}^{n} f^{p}(r_{ij})
O^{p}(i,j) \Big\},
\end{equation}


\end{subequations}
\end{small}

\end{multicols}


\vspace{-1mm}
\centerline{\rule{80mm}{0.1pt}}
\vspace{2mm}




\begin{multicols}{2}

\begin{thebibliography}{90}

\vspace{3mm}

\bibitem{lab1}LIU M L, ZHANG Y H, ZHOU X H et al. Phys. Rev. C, 2004, {\bf 70}: 14---34

\bibitem{lab2} Tinkham M. Group Theory and Quantum Mechanics. New
York: McGraw-Hill, 1964. 10---50

\bibitem{lab3} Tel T. Experimental Study and Characterization of
Chaos. Ed. Hao B. Chaos, Singapore, World Scientific, 1990. 149

\end{thebibliography}
\end{multicols}


\clearpage

\end{CJK*}
\end{document}
